\section{2-es rúd igénybevételei}

\subsection{SZTÁ}

\newpage

\subsection{Függvények}

\begin{equation*}
	\phi = \ang{30}
\end{equation*}

\begin{align*}
	N(\phi) 
	&= N_{2_x} \cdot \cos\phi + N_{2_y} \cdot \cos\left(\ang{90}-\phi\right) \\
	&= N_{2_x} \cdot \cos\phi + N_{2_y} \cdot \sin\phi \\
	N(\ang{30}) &= \n{-2837.92}
\end{align*}

\begin{align*}
	V(\phi) &= -N_{2_x} \cdot \sin\phi + N_{2_y} \cdot \sin\phi \\
	V(\ang{30}) &= \n{-760.42}
\end{align*}

\begin{align*}
	M_h(\phi) &= N_{2_x} \cdot R \left(1 - \cos\phi\right) - N_{2_y} \cdot R\sin\phi \\
	M_h(\ang{30}) &= \nmm{228125.3329}
\end{align*}

\newpage

\subsection{Normálfeszültség ábrázolása}

\subsection{Maximális normálfeszültség}

\begin{align*}
	\frac{R}{d} = 6 \Rightarrow \sigma(y) &= \frac{N}{A} + \frac{M_h}{R\cdot A} + \frac{M_h}{I_x} \cdot \frac{R\cdot y}{R+y} \\
	A &= \frac{d^2 \pi}{4} = 625\pi \\
	I_x &= \frac{d^4 \pi}{64} = \frac{390625}{4}\pi
\end{align*}

\begin{align*}
	\sigma(\frac{d}{2}) &= \mpa{16.1} \\
	\sigma(0) &= \mpa{-1.058} \\
	\sigma(-\frac{d}{2}) &= \mpa{-21.34}
\end{align*}
