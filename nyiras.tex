\section{Nyírásból adódó csúsztató feszültség}

\subsection{Függvény}

\begin{equation*}
	V_\text{max} = V_x (L+R) = \kn{-1.5765}
\end{equation*}

A csúsztató feszültséget a VISA képlettel meg lehet kapni. (Előjellel nem foglalkozunk.)
\begin{equation*}
	\tau_z(y) = \frac{V_\text{max} \cdot S(y)}{I_x \cdot a(y)}
\end{equation*}

\subsubsection{Kettébontás}
A húsvastagság változásánál felbontjuk a keresztmetszetet.

\begin{multicols}{2}
	\circled{1} $\Rightarrow 0 < y < \frac{1}{3}b$
	\begin{equation*}
		S_1(y) = A_1(y) \cdot k_1(y)
	\end{equation*}
	\begin{align*}
		A_1(y) &= \left[ 
			\frac{b}{6} 
			- \left(y - \left[\frac{b}{2} - \frac{b}{6} \right] \right)
		\right] \cdot 2b
		\\&= b^2 -2by
	\end{align*}
	\begin{equation*}
		k_1(y) = \frac{\frac{b}{2} + y}{2} = \frac{1}{4}b+\frac{1}{2}y
	\end{equation*}

	\columnbreak
	\circled{2} $\Rightarrow \frac{1}{3}b < y < \frac{1}{2}b$
	\begin{equation*}
		S_2(y) = S_1\left(\frac{2}{3}b\right) + A_2(y) \cdot k_2(y)
	\end{equation*}
	\begin{align*}
		A_2(y) &= 2\left(\frac{b}{3} \left[\left(\frac{b}{2} - \frac{b}{6}\right) - y \right] \right)
		     \\&=\frac{2}{9}b^2 - \frac{2}{3}by
	\end{align*}
	\begin{equation*}
		k_2(y) = \frac{y + \left(\frac{b}{2} - \frac{b}{6} \right)}{2}
		= \frac{1}{6}b + \frac{1}{2}y
	\end{equation*}

\end{multicols}

\begin{align*}
	&\tau_{z_1}(y) = \num{-1.776841346e-5}y^2 + \num{2.557211538e-3} \\
	&\tau_{z_2}(y) = \num{-1.776404748e-5}y^2 + \num{5.4e-3}
\end{align*}

\begin{equation*}
	\tau^{(1)}_\text{max} = \tau_{z_2}(0) = \mpa{5.4}
\end{equation*}

\newpage

\subsection{Ábrázolás}
\rule{0pt}{130pt}
\begin{center}
	\begin{tikzpicture}
		\begin{axis}[
			axis lines = left,
			xlabel={$y$},
			ylabel={$\tau\mpa{}$},
			width=\textwidth
		]
			\addplot [
				domain=\b/3:\b/2,
				red
			]
				{(-1.776841346e-5) * x^2 + 2.557211538e-3};
			\addplot [
				domain=-\b/3:\b/3,
				red
			]
				{(-1.776404748e-5) * x^2 + 5.4e-3};
			\addplot [
				domain=-\b/3:-\b/2,
				red
			]
				{(-1.776841346e-5) * x^2 + 2.557211538e-3};
			\addplot[red] coordinates{(-8, 1.420033065e-3) (-8, 4.263100961e-3)};
			\addplot[red] coordinates{(8, 1.420033065e-3) (8, 4.263100961e-3)};

			\addplot[only marks, mark=*] coordinates {(-8, 1.420033065e-3)};
			\addplot[only marks, mark=*] coordinates {(8, 1.420033065e-3)};
			\node at (0, 1.420033065e-3 - .0003) {$\num{1.420033065e-3}$};

			\addplot[only marks, mark=*] coordinates {(-8, 4.263100961e-3)};
			\addplot[only marks, mark=*] coordinates {(8, 4.263100961e-3)};
			\node at (0, 4.263100961e-3 - .0003) {$\num{4.263100961e-3}$};

			\addplot[only marks, mark=*] coordinates {(0, 5.4e-3)};
			\node at (0, 5.4e-3 - .0003) {$\num{5.4e-3}$};
		\end{axis}
	\end{tikzpicture}
\end{center}
