\documentclass{article}

\usepackage[provide=*, magyar]{babel}

\usepackage{graphicx}
\usepackage{tikz}
\usepackage{pgfplots}

\pgfplotsset{compat=1.18}
\usetikzlibrary{calc}
\usepackage{calc}

\usetikzlibrary {arrows.meta}

\usepackage{pdfpages}

\usepackage{amsmath}
\usepackage{siunitx}
\usepackage{tabularx}
\usepackage{booktabs}
\usepackage[table]{xcolor}
\usepackage{multicol}

\newcommand{\siunit}[2]{
	\SI{#1}{[#2]}
}

\newcommand{\n}[1]{
	\siunit{#1}{\newton}
}
\newcommand{\nmm}[1]{
	\siunit{#1}{\newton\mm}
}
\newcommand{\kn}[1]{
	\siunit{#1}{\kilo\newton}
}
\newcommand{\knm}[1]{
	\siunit{#1}{\kilo\newton\meter}
}
\newcommand{\mpa}[1]{
	\siunit{#1}{\mega\pascal}
}

\newcommand{\equal}[2]{
	\sum{#1} := 0 = #2
}

\newcommand{\circled}[1]{
	\raisebox{.5pt}{\textcircled{\raisebox{-.9pt} {#1}}}
}

\title{Szilárdságtan HF1}
\date{\today}
\author{Vári Gergő}

\pgfmathsetmacro\s{0.008}

\pgfmathsetmacro\L{350}
\pgfmathsetmacro\R{300}
\pgfmathsetmacro\d{50}

\pgfmathsetmacro\dones{5}
\pgfmathsetmacro\ds{\d * .05}
\pgfmathsetmacro\dthrees{4}

\newcommand{\structurecolor}{lightgray}
\newcommand{\coordcolor}{orange}
\newcommand{\normalforcecolor}{blue}
\newcommand{\sharedforcecolor}{red}
\newcommand{\reactionforcecolor}{violet}
\newcommand{\beamforcecolor}{olive}

\newcommand{\coordsize}{4pt}
\newcommand{\sizelength}{6}
\newcommand{\sizewidth}{.1}
\newcommand{\sizeyoffset}{-.5}
\newcommand{\sizeylineoffset}{-.2}

\newcommand{\forcewidth}{2}
\newcommand{\forcelength}{1.5}

\newcommand{\coords}{
	\pgfmathsetmacro\Ls{\L * \s}
	\pgfmathsetmacro\Rs{\R * \s}

	\coordinate (A) at (0, 0);
	\coordinate (B) at (2 * \Ls + \Rs, 0);
	\coordinate (C) at (\Ls + \Rs, -\Rs);
	\coordinate (D) at (\Ls + \Rs - 0.5 * \Ls, -1.5 * \Rs);
	\coordinate (G) at (\Ls, 0);
}

\newcommand{\upperpoints}{
	\fill[\coordcolor] (A) circle (\coordsize) node[above left] {$A$};
	\fill[\coordcolor] (B) circle (\coordsize) node[below right] {$B$};
	\fill[\coordcolor] (G) circle (\coordsize) node[below right] {$G$};
}

\newcommand{\midpoints}{
	\fill[\coordcolor] (G) circle (\coordsize) node[below right] {$G$};
	\fill[\coordcolor] (C) circle (\coordsize) node[below right] {$C$};
}

\newcommand{\lowerpoints}{
	\fill[\coordcolor] (C) circle (\coordsize) node[below right] {$C$};
	\fill[\coordcolor] (B) circle (\coordsize) node[below right] {$B$};
	\fill[\coordcolor] (D) circle (\coordsize) node[below left] {$D$};
}

\newcommand{\points}{
	\upperpoints
	\midpoints
	\lowerpoints
}

\newcommand{\beamone}{
	\draw[line width=\dones, \structurecolor] (A) -- (B);
	\draw[line width=\ds, \structurecolor] (G) -- +(0, \Rs * 0.5);
}

\newcommand{\beamtwo}{
	\draw[line width=\ds, \structurecolor] (G) arc (0:90:-\Rs);
}

\newcommand{\beamthree}{
	\draw[line width=\dthrees, \structurecolor] (B) -- (D);
}

\newcommand{\structure}{
	\beamone
	\beamthree
	\beamtwo
}

\newcommand{\horizontalsizes}{
	\upperhorizontalsizes

	\draw[line width=\sizewidth] (C) -- +(0, -\sizelength * 0.35);
	\draw[line width=\sizewidth] (D) -- +(0, -\sizelength * 0.15);
	\draw[line width=\sizewidth] (\Ls + \Rs, 0) -- +(0, -\sizelength * 0.5);

	\draw[line width=\sizewidth, Stealth-Stealth, ]
		(D)+(0, -\sizelength * 0.1) -- +(\Ls * .5, -\sizelength * 0.1)
		node[midway, below] {$\frac{L}{2}$};
}

\newcommand{\upperhorizontalsizes}{
	\draw[line width=\sizewidth] (A) -- +(0, \sizelength * 0.5);
	\draw[line width=\sizewidth] (G) -- +(0, \sizelength * 0.5);
	\draw[line width=\sizewidth] (\Ls + \Rs, 0) -- +(0, \sizelength * 0.5);
	\draw[line width=\sizewidth] (B) -- +(0, \sizelength * 0.5);
	
	\draw[line width=\sizewidth, Stealth-Stealth, ]
		(0, \sizelength * 0.4) -- +(\Ls, 0)
		node[midway, above] {$L$};
	\draw[line width=\sizewidth, Stealth-Stealth, ]
		(\Ls, \sizelength * 0.4) -- +(\Rs, 0)
		node[midway, above] {$R$};
	\draw[line width=\sizewidth, Stealth-Stealth, ]
		(\Ls + \Rs, \sizelength * 0.4) -- +(\Ls, 0)
		node[midway, above] {$L$};
}

\newcommand{\verticalsizes}{
	\draw[line width=\sizewidth] (\sizeyoffset, \Rs * 0.5) -- +(-\sizelength * 0.1, 0);
	\draw[line width=\sizewidth] (\sizeyoffset, 0) -- +(-\sizelength * 0.1, 0);
	\draw[line width=\sizewidth] (\sizeyoffset, -\Rs) -- +(-\sizelength * 0.1, 0);
	\draw[line width=\sizewidth] (\sizeyoffset, -\Rs -\Rs * 0.5) -- +(-\sizelength * 0.1, 0);

	\draw[line width=\sizewidth, Stealth-Stealth, ]
		(\sizeyoffset + \sizeylineoffset, \Rs * 0.5) -- +(0, -\Rs * 0.5)
		node[midway, left] {$\frac{R}{2}$};
	\draw[line width=\sizewidth, Stealth-Stealth, ]
		(\sizeyoffset + \sizeylineoffset, 0) -- +(0, -\Rs)
		node[midway, left] {$R$};
	\draw[line width=\sizewidth, Stealth-Stealth, ]
		(\sizeyoffset + \sizeylineoffset, -\Rs) -- +(0, -\Rs * 0.5)
		node[midway, left] {$\frac{R}{2}$};
}

\newcommand{\sizes}{
	\horizontalsizes
	\verticalsizes
}

\newcommand{\normalforces}{
	\draw[line width=\forcewidth, \normalforcecolor, -Stealth] 
		(D) -- +(-\forcelength, 0)
		node[midway, above] {$F_2$};
	\draw[line width=\forcewidth, \normalforcecolor, -Stealth] 
		(\Ls, \Rs * 0.485) -- +(-\forcelength, 0)
		node[midway, above] {$F_1$};
}

\newcommand{\sharedforces}{
	\fill[\sharedforcecolor, opacity=.4] (A) rectangle +(\Ls + \Rs, \Rs * 0.3);
	\draw[line width=.2, \sharedforcecolor, -Stealth] 
		(1, \Rs * 0.25) -- (1, .2)
		node[midway, left] {$p$};
}

\newcommand{\reactionforces}{
	\draw[line width=\forcewidth, \reactionforcecolor, -Stealth] 
		(0, -\forcelength) -- (A)
		node[midway, right] {$A_y$};
	\draw[line width=\forcewidth, \reactionforcecolor, -Stealth] 
		(-\forcelength, 0) -- (A)
		node[near start, left, below] {$A_x$};

	\draw[line width=\forcewidth, \reactionforcecolor, -Stealth] 
		(B) -- +(\forcelength, 0)
		node[below right] {$B_x$};
	\draw[line width=\forcewidth, \reactionforcecolor, -Stealth] 
		(2*\Ls + \Rs, -\forcelength) -- (B)
		node[near start, right] {$B_y$};
}

\newcommand{\forces}{
	\normalforces
	\sharedforces
	\reactionforces
}

\newcommand{\convention}{
	\draw[-Stealth] 
		(1.5 * \Ls + \Rs, 1.5 * \Rs) -- +(1, 0)
		node [below] {$x$};
	\draw[-Stealth] 
		(1.5 * \Ls + \Rs, 1.5 * \Rs) -- +(0, 1)
		node [left] {$y$};

	\draw[-Stealth]
		(2 * \Ls + \Rs, 1.75 * \Rs) arc (0:180:-.5)
		node [midway, above] {$+$};
}

\newcommand{\szta}{
	\begin{figure}[hbt!]
		\centering
		\begin{tikzpicture}
			\coords
			
			\convention
			\sizes
			\forces
			\structure
			\points
		\end{tikzpicture}
		\caption{SZTÁ}
	\end{figure}
}

\newcommand{\beamoneforces}{
	\draw[line width=\forcewidth, \beamforcecolor, -Stealth] 
		(B) -- +(\forcelength, 0)
		node[near end, below] {$N_{1_x}$};
	\draw[line width=\forcewidth, \beamforcecolor, -Stealth] 
		(2 * \Ls + \Rs, \forcelength) -- (B)
		node[midway, right] {$N_{1_y}$};

	\draw[line width=\forcewidth, \beamforcecolor, -Stealth] 
		(G) -- +(\forcelength, 0)
		node[near end, below] {$N_{2_x}$};
	\draw[line width=\forcewidth, \beamforcecolor, -Stealth] 
		(G) -- +(0, -\forcelength)
		node[midway, right] {$N_{2_y}$};
}

\newcommand{\sztaone}{
	\begin{figure}[hbt!]
		\centering
		\begin{tikzpicture}
			\coords
			
			\convention

			\upperhorizontalsizes
			\sharedforces
		\draw[line width=\forcewidth, \normalforcecolor, -Stealth] 
			(\Ls, \Rs * 0.485) -- +(-\forcelength, 0)
			node[midway, above] {$F_1$};
			\beamone
			\beamoneforces
			\upperpoints
		\end{tikzpicture}
		\caption{SZTÁ}
	\end{figure}
}

\newcommand{\beamtwoforces}{
	\draw[line width=\forcewidth, \beamforcecolor, -Stealth] 
		(\forcelength + \Ls, 0)-- (G)
		node[near start, below] {$N_{2_x}$};
	\draw[line width=\forcewidth, \beamforcecolor, -Stealth] 
		(G) -- +(0, \forcelength)
		node[midway, right] {$N_{2_y}$};

	\draw[line width=\forcewidth, \beamforcecolor, -Stealth] 
		(C) -- +(\forcelength, 0)
		node[near end, above] {$N_{2_x}$};
	\draw[line width=\forcewidth, \beamforcecolor, -Stealth] 
		(C) -- +(0, -\forcelength)
		node[midway, left] {$N_{2_y}$};
}

\newcommand{\beamthreeforces}{
	\draw[line width=\forcewidth, \beamforcecolor, -Stealth] 
		(C) -- +(-\forcelength, 0)
		node[left] {$N_{2_x}$};
	\draw[line width=\forcewidth, \beamforcecolor, -Stealth] 
		(C) -- +(0, \forcelength)
		node[above] {$N_{2_y}$};

	\draw[line width=\forcewidth, \beamforcecolor, -Stealth] 
		(B) -- +(\forcelength, 0)
		node[right] {$N_{3_x}$};
	\draw[line width=\forcewidth, \beamforcecolor, -Stealth] 
		(B) -- +(0, \forcelength)
		node[above] {$N_{3_y}$};
}

\newcommand{\sztatwo} {
	\begin{figure}[hbt!]
		\centering
		\begin{tikzpicture}
			\coords
			
			\verticalsizes
			\beamtwoforces
			\beamtwo
			\midpoints
		\end{tikzpicture}
		\caption{SZTÁ}
	\end{figure}
}

\newcommand{\sztathree} {
	\begin{figure}[hbt!]
		\centering
		\begin{tikzpicture}
			\coords
			
			\sizes
			\beamthreeforces

			\draw[line width=\forcewidth, \normalforcecolor, -Stealth] 
				(D) -- +(-\forcelength, 0)
				node[midway, above] {$F_2$};

			\beamthree
			\lowerpoints
		\end{tikzpicture}
		\caption{SZTÁ}
	\end{figure}
}

\newcommand{\sztab} {
	\begin{figure}[hbt!]
		\centering
		\begin{tikzpicture}
			\coords
			
			\draw (0, 0) circle (.5) node {$B$};

			\draw[line width=\forcewidth, \beamforcecolor, -Stealth] 
				(-.5, 0) -- +(-\forcelength, 0)
				node[midway, above] {$N_{1_x}$};
			\draw[line width=\forcewidth, \beamforcecolor, -Stealth] 
				(.5, 0) -- +(\forcelength, 0)
				node[midway, above] {$N_{3_x}$};
			\draw[line width=\forcewidth, \reactionforcecolor, -Stealth] 
				(0, 0.5) -- +(0, \forcelength)
				node[above] {$B_y$};
			\draw[line width=\forcewidth, \beamforcecolor, -Stealth] 
				(.7, 0.5) -- +(0, \forcelength)
				node[above] {$N_{1_y}$};
			\draw[line width=\forcewidth, \beamforcecolor, -Stealth] 
				(0, -0.5) -- +(0, -\forcelength)
				node[right] {$N_{3_y}$};

		\end{tikzpicture}
		\caption{SZTÁ}
	\end{figure}
}


\begin{document}
	\pagenumbering{gobble}
	
	\input{exercise}

	\maketitle
	\rule{0pt}{50pt}
	\begin{figure}[hbt!]
		\centering
		\includegraphics[scale=1.75]{./images/cauchy_stress_components.png}
		\caption{Cauchy feszültségi tenzor}
	\end{figure}

	\newpage
	\pagenumbering{arabic}
	
	\section{Reakció komponensek}

\subsection{Léptékhelyes ábra}

\pgfmathsetmacro\s{0.01}

\pgfmathsetmacro\L{350}
\pgfmathsetmacro\R{300}


\pgfmathsetmacro\d{50}

\pgfmathsetmacro\dones{5}
\pgfmathsetmacro\ds{\d * .05}
\pgfmathsetmacro\dthrees{4}

\newcommand{\coords}{
	\pgfmathsetmacro\Ls{\L * \s}
	\pgfmathsetmacro\Rs{\R * \s}

	\coordinate (A) at (0, 0);
	\coordinate (B) at (2 * \Ls + \Rs, 0);
	\coordinate (C) at (\Ls + \Rs, -\Rs);
	\coordinate (D) at (\Ls + \Rs - 0.5 * \Ls, -1.5 * \Rs);
	\coordinate (G) at (\Ls, 0);
}
\newcommand{\points}{
	\fill[red] (A) circle (2pt) node[above left] {A};
	\fill[red] (B) circle (2pt) node[below right] {B};
	\fill[red] (C) circle (2pt) node[below right] {C};
	\fill[red] (D) circle (2pt) node[below left] {D};
	\fill[red] (G) circle (2pt) node[above] {G};
}
\newcommand{\structure}{
	\draw[line width=\dones] (A) -- (B);
	\draw[line width=\dthrees] (B) -- (D);

	\draw[line width=\ds] (G) arc (0:90:-\Rs);
}
\newcommand{\sizes}{
	
}

\begin{center}
	\begin{tikzpicture}
		\coords
		
		\structure
		\sizes
		\points
	\end{tikzpicture}
\end{center}

\subsection{SZTÁ}

\newpage

\subsection{Egyensúlyi képletek}

\begin{align*}
	&\equal{F_x}{A_x - F_1 - F_2} \\
	&\equal{F_y}{A_y + B_y - p(L+R)} \\
	&\equal{M^A}
	{B_y(2L+R) + F_1 \frac{R}{2} - F_2 \left(R+\frac{R}{2}\right) - p\frac{(L+R)^2}{2}}
\end{align*}

\begin{align*}
	&A_x = F_1 + F_2 = \kn{6} \\
	&B_y 
		= F_2 \left(R + \frac{R}{2}\right) - F_1\frac{R}{2} + p\frac{(L+R)^2}{2} 
		= \kn{1.85} \\
	&A_y = p(L+R) - B_y = \kn{1.074} \\
\end{align*}

\begin{align*}
	&|\textbf{A}| = \kn{6.1} \\
	&|\textbf{B}| = \kn{1.85} 
\end{align*}

	\newpage

	\section{Csuklók és rudak}

\subsection{}

\subsubsection{SZTÁ}

\subsubsection{Egyensúlyi képletek}
\begin{align*}
    &\equal{F_x}{A_x - F_1 + N_{2_x} + N_{1_x}} \\
    &\equal{F_y}{A_y - p(L+R) - N_{2_y} - N_{1_Y}} \\
    &\equal{M^B}{N_{2_y}(L+R) + F_1 \frac{R}{2} + p(L+R)\left(L+\frac{L+R}{2}\right) \nonumber \\
    &\quad - A_y (2L+R)}
\end{align*}

\begin{equation*}
	N_{2_y} = \frac{A_y{2L+R} - F_1 \frac{R}{2} - p(L+R)\left(L+\frac{L+R}{2}\right)}{L+R} = \kn{-2.0775}
\end{equation*}

\newpage

\subsection{}

\subsubsection{SZTÁ}

\subsubsection{Egyensúlyi képletek}
\begin{align*}
    &\equal{F_x}{- N_{2_x} + N_{2_x}} \\
    &\equal{F_y}{- N_{2_y} + N_{2_y}} \\
    &\equal{M}{0}
\end{align*}

\newpage

\subsection{}
\subsubsection{SZTÁ}
\subsubsection{Egyensúlyi képletek}
\begin{align*}
    &\equal{F_x}{- F_2 + N_{3_x} - N_{2_x}} \\
    &\equal{F_y}{N_{2_y} + N_{3_y}} \\
    &\equal{M^C}{-F_2 \frac{R}{2} - N_{3_x} (R) + N_{3_y} (L)}
\end{align*}

\newpage

\subsection{B pont}

\subsubsection{SZTÁ}

\subsubsection{Egyensúlyi képletek}
\begin{align*}
    &\equal{F_x}{-N_{1_x} + N_{3_x}} \\
    &\equal{F_y}{N_{1_y} + B_y - N_{3_y}}
\end{align*}

\newpage

\subsection{Összegzés}
\begin{align*}
	N_1 &= &\begin{bmatrix}
		0.92375 \\
		0.2265
	\end{bmatrix} \kn{}\\
	N_2 &= &\begin{bmatrix}
		-2.0775 \\
		-2.0775
	\end{bmatrix} \kn{}\\
	N_3 &= &\begin{bmatrix}
		0.92375 \\
		2.0775
	\end{bmatrix} \kn{}\\
\end{align*}

	\newpage

	\section{1-es rúd igénybevételei}

\sztaone

\newcommand{\ncolor}{orange}
\newcommand{\vcolor}{green}
\newcommand{\mcolor}{cyan}

\subsection{Függvények}
{\footnotesize
	\begin{center}
		\setlength{\aboverulesep}{0pt}
		\setlength{\belowrulesep}{0pt}
		\setlength{\extrarowheight}{.75ex}
		\begin{tabular}{rccc}
			\toprule
			\rowcolor{lightgray}
			$x$
			&$0 < x < L$
			&$L < x < L+R$
			&$L+R < x < 2L+R$ \\

			\toprule

			\rowcolor{\ncolor}
			$N$
			&$-A_x$
			&$-A_x - N_{2_x} + F_1$
			&$-A_x - N_{2_x} + F_1$ \\

			\midrule

			\rowcolor{\vcolor}
			$V$
			&$-A_y+px$
			&$px -A_y + N_{2_y}$
			&$p(L+R) - A_y + N_{2_y}$ \\

			\midrule

			\rowcolor{\mcolor}
			$M_h$
			&$
			-A_y x + p \frac{x^2}{2}
			$
			&$
			\begin{array}{c}
				-A_y x + p \frac{x^2}{2} \\
				+ N_{2_y}(x - L) + F_1 \frac{R}{2}
			\end{array}
			$
			&$
			\begin{array}{c}
				\\
				-A_y x + p (L+R) \left( L-\frac{L+R}{2} \right) \\
				+ N_{2_y}(x - L) + F_1 \frac{R}{2} \\
				\\
			\end{array}$ \\

			\bottomrule
		\end{tabular}
	\end{center}
}

\newpage

\subsection{Ábrázolás}

\pgfmathsetmacro\Ax{6}
\pgfmathsetmacro\Ay{1.074}
\pgfmathsetmacro\p{0.0045}
\pgfmathsetmacro\Ntwox{-2.0775}
\pgfmathsetmacro\Ntwoy{\Ntwox}
\pgfmathsetmacro\Fone{3}

\begin{center}
	\begin{tikzpicture}
		\coords

		\upperhorizontalsizes

		\draw[opacity=.2] (A) -- +(0, -15);
		\draw[opacity=.2] (G) -- +(0, -15);
		\draw[opacity=.2] (\Ls + \Rs, 0) -- +(0, -15);
		\draw[opacity=.2] (B) -- +(0, -15);

		\sharedforces
		\draw[line width=\forcewidth, \normalforcecolor, -Stealth] 
			(\Ls, \Rs * 0.485) -- +(-\forcelength, 0)
			node[midway, above] {$F_1$};
		\beamone
		\beamoneforces
		\upperpoints

		\begin{axis}[
			at={(0, -1500)},
			height={100},
			width={272.5},
			xlabel={$x$},
			ylabel={$N(x) \kn{}$},
			ymin=-8, ymax=0,
			axis lines=left,
		]
			\addplot [
				domain=0:\L,
				\ncolor
			]
				{-\Ax};
			\addplot [
				domain=\L:\L+\R,
				\ncolor
			]
				{-\Ax-\Ntwox};
			\addplot [
				domain=\L+\R:\L+\L+\R,
				\ncolor
			]
				{-\Ax-\Ntwox+\Fone};
		\end{axis}

		\begin{axis}[
			at={(0, -2500)},
			height={100},
			width={272.5},
			xlabel={$x$},
			ylabel={$V(x) \kn{}$},
			ymin=-3, ymax=3,
			axis lines=left,
		]
			\addplot [
				domain=0:\L,
				\vcolor
			]
				{-\Ay + \p *x};
			\addplot [
				domain=\L:\L+\R,
				\vcolor
			]
				{-\Ay + \p*x + \Ntwoy};
			\addplot [
				domain=\L+\R:\L+\L+\R,
				\vcolor
			]
				{-\Ay + \p*(\L+\R) + \Ntwoy};
		\end{axis}

		\begin{axis}[
			at={(0, -4500)},
			height={100},
			width={272.5},
			xlabel={$x$},
			ylabel={$M_h(x) \knm{}$},
			ymin=-300, ymax=400,
			axis lines=left,
		]
			\addplot [
				domain=0:\L,
				\mcolor
			]
				{-\Ay *x + \p *(x^2/2)};
			\addplot [
				domain=\L:\L+\R,
				\mcolor
			]
				{-\Ay *x + \p *(x^2/2) + \Ntwoy * (x - \L) + \Fone * (\R / 2)};
			\addplot [
				domain=\L+\R:2*\L+\R,
				\mcolor
			]
				{-\Ay *x + \p *(\L+\R) *(x - (\L+\R)/2) + \Ntwoy * (x - \L) + \Fone * (\R / 2)};
		\end{axis}
	\end{tikzpicture}
\end{center}

	\newpage
	
	\section{Méretezés}

\subsection{Veszélyes keresztmetszet}
Ugyan $V(x)$ nem metszi az $x$ tengelyt és ezért $M_h(x)$ maximuma nem triviális de ez rajzolás után könnyen megállapítható.
\begin{equation*}
	M_{h_\text{max}} = M_h(L) = \knm{0.35}
\end{equation*}

\subsection{Keresztmetszeti tényező}

\begin{align*}
	&\left|\frac{M_h}{K_x}\right| = \sigma \\
	&K_{x_\text{min}} = \frac{M_{h_\text{max}}}{\sigma_\text{meg}} = \siunit{3.5}{\cm^3}
\end{align*}

Tiszta hajlításra méretezve a keresztmetszetet:
\begin{align*}
	&\sigma = \frac{M_{h_\text{max}}}{I_x} y \\
	&e = \frac{b}{2}
\end{align*}

\begin{align*}
	&K_{x_\text{min}} 
	= \frac{I_x}{e} = \frac{65}{243} b^3 \\
	&b = 23.66 \approx \siunit{24}{\mm}
\end{align*}

	\section{Helyettesítés U-szelvénnyel}
Egyszerűen megtalálható az adott táblázatban.
\begin{center}\Large{166}\end{center}

	\section{U-szelvény ellenőrzése normálerő hatására}

\subsection{Ellenőrzés}
\begin{equation*}
	N(L) = \kn{-6} 
\end{equation*}

\begin{align*}
	\sigma &= \frac{N}{A} + \frac{M_h}{I_x} y \\
	A &= 2b \cdot b - f \cdot b = \frac{10}{9}b^2 \\
	I_x &= \frac{65}{486} b^4
\end{align*}

\begin{gather*}
	\sigma^{(1)}_\text{max} = \sigma_e 
	= \frac{N}{A} + \frac{M_{h_\text{max}}}{I_x} e 
	= \mpa{-94.66} \\
	\left|\sigma_e\right| > \sigma_\text{meg} \Rightarrow b^* = b
\end{gather*}

\subsection{Normálerő ábrázolása}

	\newpage

	\section{Nyírásból adódó csúsztató feszültség}

\subsection{Függvény}

\subsection{Ábrázolás}

	\newpage

	\section{2-es rúd igénybevételei}

\subsection{SZTÁ}

\newpage

\subsection{Függvények}

\newpage

\subsection{Normálfeszültség ábrázolása}

\subsection{Maximális normálfeszültség}

	\newpage
	
	\section{3-as rúd hajlítása}

\subsection{Zérus és $y_3$ tengely szöge}

\subsection{Maximális normálfeszültség}

\end{document}
